% not compatible with [scale=?]


%----------------------------------------------------  
  
    \begin{tikzpicture}
        \begin{scope}[local bounding box=WMT]
            \draw[->,thick] (0.4,0) to (11.5,0);
            \draw[->,thick] (0.4,-0) to (0.4,4.3);
            \draw[thick] (0.4,2) to (0.6,2);
            \draw[thick] (0.4,4) to (0.6,4);
            \node[font=\scriptsize] at (0,2) {10};
            \node[font=\scriptsize] at (0,4) {20};

            % 2015
            \node[minimum width=0.7cm,thick,minimum height=7*0.2cm,fill=blue!30!white,inner sep=0pt,outer sep=0pt,anchor=south west] (smt2015) at (1.5*0.7,0.5pt) {};
            \node[minimum width=0.7cm,thick,minimum height=2*0.2cm,fill=red!30!white,inner sep=0pt,outer sep=0pt,anchor=south west] (nmt2015) at (smt2015.south east) {};
            \node[font=\scriptsize,anchor=north] () at ([yshift=-0.2em]smt2015.south east) {2015};
            % 2016
            \node[minimum width=0.7cm,thick,minimum height=3*0.2cm,fill=blue!30!white,inner sep=0pt,outer sep=0pt,anchor=south west] (smt2016) at ($(nmt2015.south east)+(0.7,0)$) {};
            \node[minimum width=0.7cm,thick,minimum height=8*0.2cm,fill=red!30!white,inner sep=0pt,outer sep=0pt,anchor=south west] (nmt2016) at (smt2016.south east) {};
            \node[font=\scriptsize,anchor=north] () at ([yshift=-0.2em]smt2016.south east) {2016};
            % 2017
            \node[minimum width=0.7cm,thick,minimum height=3*0.2cm,fill=blue!30!white,inner sep=0pt,outer sep=0pt,anchor=south west] (smt2017) at ($(nmt2016.south east)+(0.7,0)$) {};
            \node[minimum width=0.7cm,thick,minimum height=13*0.2cm,fill=red!30!white,inner sep=0pt,outer sep=0pt,anchor=south west] (nmt2017) at (smt2017.south east) {};
            \node[font=\scriptsize,anchor=north] () at ([yshift=-0.2em]smt2017.south east) {2017};
            % 2018
            \node[minimum width=0.7cm,thick,minimum height=0cm,draw,fill=blue!30!white,inner sep=0pt,outer sep=0pt,anchor=south west] (smt2018) at ($(nmt2017.south east)+(0.7,0)$) {};
            \node[minimum width=0.7cm,thick,minimum height=14*0.2cm,fill=red!30!white,inner sep=0pt,outer sep=0pt,anchor=south west] (nmt2018) at (smt2018.south east) {};
            \node[font=\scriptsize,anchor=north] () at ([yshift=-0.2em]smt2018.south east) {2018};
             % 2019
            \node[minimum width=0.7cm,thick,minimum height=0cm,draw,fill=blue!30!white,inner sep=0pt,outer sep=0pt,anchor=south west] (smt2019) at ($(nmt2018.south east)+(0.7,0)$) {};
            \node[minimum width=0.7cm,thick,minimum height=21*0.2cm,fill=red!30!white,inner sep=0pt,outer sep=0pt,anchor=south west] (nmt2019) at (smt2019.south east) {};
            \node[font=\scriptsize,anchor=north] () at ([yshift=-0.2em]smt2019.south east) {2019};
        \end{scope}

        % legend
        \ExtractX{$(nmt2015.west)$}
        \ExtractY{$(WMT.north)$}
        \node[minimum width=0.7cm,rectangle,fill=blue!30!white,anchor=north west,label={[label distance=1pt,font=\scriptsize]0:统计机器翻译}] () at (\XCoord,\YCoord) {};
        \ExtractX{$(nmt2017.west)$}
        \node[minimum width=0.7cm,rectangle,fill=red!30!white,anchor=north west,label={[label distance=1pt,font=\scriptsize]0:神经机器翻译}] () at (\XCoord,\YCoord) {};

  
       % \node[font=\normalsize,rotate=90] () at ([xshift=-1em]WMT.west) {数量};
       \node[font=\normalsize] () at (0.4,4.5) {数量};
        \node[font=\normalsize] () at (11.5,-0.3) {年份};
        
        
    \end{tikzpicture}