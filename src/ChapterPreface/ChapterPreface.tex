% !Mode:: "TeX:UTF-8"
% !TEX encoding = UTF-8 Unicode

%----------------------------------------------------------------------------------------
% 机器翻译:统计建模与深度学习方法
% Machine Translation: Statistical Modeling and Deep Learning Methods
%
% Copyright 2020
% 肖桐(xiaotong@mail.neu.edu.cn) 朱靖波 (zhujingbo@mail.neu.edu.cn)
%----------------------------------------------------------------------------------------

\renewcommand\figurename{图}

%----------------------------------------------------------------------------------------
%	PREFACE
%----------------------------------------------------------------------------------------

{\color{white} 空}
\vspace{0.5em}
\begin{center}
{\Huge \bfnew{导\ \ \ \ 读}}
\end{center}
\vspace{2em}

\begin{spacing}{1.18}

让计算机进行自然语言的翻译是人类长期的梦想,也是人工智能的终极目标之一。自上世纪九十年代起,机器翻译迈入了基于统计建模的时代,发展到今天,深度学习等机器学习方法已经在机器翻译中得到了大量的应用,取得了令人瞩目的进步。

在这个时代背景下,对机器翻译的模型、方法和实现技术进行深入了解是自然语言处理领域研究者和实践者所渴望的。本书全面回顾了近三十年内机器翻译的技术发展历程,并围绕统计建模和深度学习两个主题对机器翻译的技术方法进行了全面介绍。在写作中,笔者力求用朴实的语言和简洁的实例阐述机器翻译的基本模型和方法,同时对相关的技术前沿进行讨论。本书可以供计算机相关专业高年级本科生及研究生学习之用,也可以作为自然语言处理,特别是机器翻译领域相关研究人员的参考资料。

本书共分为七个章节,章节的顺序参考了机器翻译技术发展的时间脉络,同时兼顾了机器翻译知识体系的内在逻辑。各章节的主要内容包括:

\begin{itemize}
\vspace{0.5em}
\item 第一章:机器翻译简介
\vspace{0.5em}
\item 第二章:词法、语法及统计建模基础
\vspace{0.5em}
\item 第三章:基于词的机器翻译模型
\vspace{0.5em}
\item 第四章:基于短语和句法的机器翻译模型
\vspace{0.5em}
\item 第五章:人工神经网络和神经语言建模
\vspace{0.5em}
\item 第六章:神经机器翻译模型
\vspace{0.5em}
\item 第七章:神经机器翻译实战 \ \dash \ 参加一次比赛
\vspace{0.5em}
\end{itemize}

其中,第一章是对机器翻译的整体介绍。第二章和第五章是对统计建模和深度学习方法的介绍,分别建立了两个机器翻译范式的基础知识体系 \ \dash \ 统计机器翻译和神经机器翻译。统计机器翻译部分(第三、四章)涉及早期的基于单词的翻译模型,以及本世纪初流行的基于短语和句法的翻译模型。神经机器翻译(第六、七章)代表了当今机器翻译的前沿,内容主要涉及了基于端到端表示学习的机器翻译建模方法。特别地,第七章对一些最新的神经机器翻译方法进行了讨论,为相关科学问题的研究和实用系统的开发提供了可落地的思路。图\ref{fig:preface}展示了本书各个章节及核心概念之间的关系。

{\red 用最简单的方式阐述机器翻译的基本思想}是笔者所期望达到的目标。但是,书中不可避免会使用一些形式化定义和算法的抽象描述,因此,笔者尽所能通过图例进行解释(本书共320张插图)。不过,本书所包含的内容较为广泛,难免会有疏漏,望读者海涵,并指出不当之处。

%-------------------------------------------
\begin{figure}[htp]
\centering
\centering
% !Mode:: "TeX:UTF-8"
% !TEX encoding = UTF-8 Unicode

\begin{tikzpicture}

\tikzstyle{secnode} =[font=\scriptsize,minimum height=4.0em,minimum width=22em,draw,thick,fill=white,drop shadow]
\tikzstyle{conceptnode} =[font=\scriptsize,minimum height=1.5em,minimum width=5em]
\tikzstyle{conceptnodesmall} =[font=\scriptsize,minimum height=1.0em,minimum width=4.4em]

% section 1
\node [secnode,anchor=south west,minimum width=22.5em,red,fill=white] (sec1) at (0,0) {};
\node [anchor=north] (sec1label) at ([yshift=-0.2em]sec1.north) {\small{机器翻译简介}};
\node [anchor=north west,draw=red,thick,fill=white,rounded corners] (sec1title) at ([xshift=-0.3em,yshift=0.3em]sec1.north west) {{\footnotesize\bfnew{\color{red} 第一章}}};
\node [conceptnode,anchor=south west,fill=red!15,thin] (sec1box1) at ([xshift=0.5em,yshift=0.5em]sec1.south west) {\footnotesize{发展历史}};
\node [conceptnode,anchor=west,fill=red!15,thin] (sec1box2) at ([xshift=0.5em]sec1box1.east) {\footnotesize{评价方法}};
\node [conceptnode,anchor=west,fill=red!15,thin] (sec1box3) at ([xshift=0.5em]sec1box2.east) {\footnotesize{应用情况}};
\node [conceptnode,anchor=west,fill=red!15,thin] (sec1box4) at ([xshift=0.5em]sec1box3.east) {\footnotesize{系统\&数据}};


% section 2
\node [secnode,anchor=south,blue,fill=white] (sec2) at ([xshift=-6.5em,yshift=3em]sec1.north) {};
\node [anchor=north] (sec2label) at (sec2.north) {\small{词法、语法及统计建模基础}};
\node [anchor=north west,draw=blue,thick,fill=white,rounded corners] (sec2title) at ([xshift=-0.3em,yshift=0.3em]sec2.north west) {{\footnotesize\bfnew{\color{blue} 第二章}}};
\node [conceptnode,anchor=south west,fill=ublue!15,thin,minimum width=4em,align=left] (sec2box1) at ([xshift=0.5em,yshift=0.4em]sec2.south west) {\tiny{概率论与}\\\tiny{统计建模基础}};
\node [anchor=west,draw,dotted,thick,minimum height=2em,minimum width=16.6em,align=left] (sec2box2) at ([xshift=0.3em]sec2box1.east) {};
\node [conceptnodesmall,minimum width=5em,anchor=south west,fill=blue!15,thin] (sec2box3) at ([xshift=0.4em,yshift=0.3em]sec2box2.south west) {\scriptsize{中文分词}};
\node [conceptnodesmall,minimum width=5em,anchor=west,fill=blue!15,thin] (sec2box4) at ([xshift=0.4em]sec2box3.east) {\scriptsize{$n$元语法模型}};
\node [conceptnodesmall,minimum width=5em,anchor=west,fill=blue!15,thin] (sec2box5) at ([xshift=0.4em]sec2box4.east) {\scriptsize{句法分析}};


\draw [->,very thick] ([xshift=-1em,yshift=0.2em]sec1.north) .. controls +(north:2.5em) and +(south:2.5em) .. ([xshift=-3em,yshift=-0.2em]sec2.south);

% section 5
\node [secnode,anchor=south,orange,fill=white] (sec5) at ([xshift=7em,yshift=10em]sec1.north) {};
\node [anchor=north] (sec5label) at (sec5.north) {\small{人工神经网络和神经语言建模}};
\node [anchor=north west,draw=orange,thick,fill=white,rounded corners] (sec5title) at ([xshift=-0.3em,yshift=0.3em]sec5.north west) {{\footnotesize\bfnew{\color{orange} 第五章}}};
\node [conceptnode,minimum width=4em,anchor=south west,fill=orange!15,thin,align=left] (sec5box1) at ([xshift=0.5em,yshift=0.4em]sec5.south west) {\tiny{线性代数基础}\\\tiny{与感知机}};
\node [conceptnode,minimum width=4em,anchor=west,fill=orange!15,thin,align=left] (sec5box2) at ([xshift=0.2em]sec5box1.east) {\tiny{多层神经网络}\\\tiny{与实现方法}};
\node [conceptnode,minimum width=4em,anchor=west,fill=orange!15,thin,align=left] (sec5box3) at ([xshift=0.2em]sec5box2.east) {\tiny{模型训练}\\\tiny{(反向传播)}};
\node [conceptnode,minimum width=4em,anchor=west,fill=purple!15,thin,align=left] (sec5box4) at ([xshift=0.2em]sec5box3.east) {\tiny{神经语言模型}\\\tiny{(FNN等)}};
\node [conceptnode,minimum width=4em,anchor=west,fill=purple!15,thin,align=left] (sec5box5) at ([xshift=0.2em]sec5box4.east) {\tiny{表示学习与}\\\tiny{预训练模型}};
\node [draw,dotted,thick,inner sep=1pt] [fit = (sec5box4) (sec5box5)] (pretrainbox) {};


\draw [->,very thick] ([yshift=-9.8em]sec5.south) -- ([yshift=-0.2em]sec5.south);
\draw [->,thick,dotted] ([xshift=0.2em,yshift=1em]sec2.east) .. controls +(east:3em) and +(south:4em) .. ([xshift=3em,yshift=-0.0em]pretrainbox.south);

% section 3
\node [secnode,anchor=south,ugreen,fill=white] (sec3) at ([yshift=10em]sec2.north) {};
\node [anchor=north] (sec3label) at (sec3.north) {\small{基于词的机器翻译模型}};
\node [anchor=north west,draw=ugreen,thick,fill=white,rounded corners] (sec3title) at ([xshift=-0.3em,yshift=0.3em]sec3.north west) {{\footnotesize\bfnew{\color{ugreen} 第三章}}};
\node [conceptnode,minimum width=4em,anchor=south west,fill=ublue!15,thin,align=left] (sec3box1) at ([xshift=0.5em,yshift=0.4em]sec3.south west) {\tiny{机器翻译的统计}\\\tiny{描述(实例)}};
\node [conceptnode,minimum width=4em,anchor=west,fill=green!20,thin,align=left] (sec3box2) at ([xshift=0.2em]sec3box1.east) {\tiny{噪声信道模型}\\\tiny{与生成式模型}};
\node [conceptnode,minimum width=4em,anchor=west,fill=green!20,thin,align=left] (sec3box3) at ([xshift=0.2em]sec3box2.east) {\tiny{IBM模型、隐}\\\tiny{马尔可夫模型}};
\node [conceptnode,minimum width=4em,anchor=west,fill=green!20,thin,align=left] (sec3box4) at ([xshift=0.2em]sec3box3.east) {\tiny{\hspace{0.9em}参数学习}\\\tiny{=优化}};
\node [conceptnode,minimum width=3.8em,anchor=west,fill=green!20,thin,align=left,minimum height=2em,inner sep=2pt] (sec3box5) at ([xshift=0.2em]sec3box4.east) {\scriptsize{EM算法}};

\draw [->,very thick] ([yshift=0.2em,xshift=-3em]sec2.north) -- ([yshift=-0.2em,xshift=-3em]sec3.south);

% section 4
\node [secnode,anchor=south,ugreen,fill=white] (sec4) at ([yshift=3em]sec3.north) {};
\node [anchor=north] (sec4label) at (sec4.north) {\small{基于短语和句法的机器翻译模型}};
\node [anchor=north west,draw=ugreen,thick,fill=white,rounded corners] (sec4title) at ([xshift=-0.3em,yshift=0.3em]sec4.north west) {{\footnotesize\bfnew{\color{ugreen} 第四章}}};
\node [conceptnode,minimum width=4em,anchor=south west,fill=ublue!15,thin,align=left] (sec4box1) at ([xshift=0.5em,yshift=0.4em]sec4.south west) {\tiny{判别式模型与}\\\tiny{最小错误率训练}};
\node [conceptnode,minimum width=4em,anchor=west,fill=green!20,thin,align=left] (sec4box2) at ([xshift=0.2em]sec4box1.east) {\tiny{基于翻译推导}\\\tiny{的建模}};
\node [conceptnode,minimum width=4em,anchor=west,fill=green!20,thin,align=left] (sec4box3) at ([xshift=0.2em]sec4box2.east) {\tiny{短语及句法}\\\tiny{翻译规则抽取}};
\node [conceptnode,minimum width=4em,anchor=west,fill=green!20,thin,align=left,minimum height=2em,inner sep=2pt] (sec4box4) at ([xshift=0.2em]sec4box3.east) {\scriptsize{调序模型}};
\node [conceptnode,minimum width=3.8em,anchor=west,fill=green!20,thin,align=left,minimum height=2em,inner sep=2pt] (sec4box5) at ([xshift=0.2em]sec4box4.east) {\scriptsize{解码}};

\draw [->,very thick] ([yshift=0.2em,xshift=-3em]sec3.north) -- ([yshift=-0.2em,xshift=-3em]sec4.south);

% section 6
\node [secnode,anchor=south,purple,fill=white] (sec6) at ([yshift=19em]sec5.north) {};
\node [anchor=north] (sec6label) at (sec6.north) {\small{神经机器翻译模型}};
\node [anchor=north west,draw=purple,thick,fill=white,rounded corners] (sec6title) at ([xshift=-0.3em,yshift=0.3em]sec6.north west) {{\footnotesize\bfnew{\color{purple} 第六章}}};
\node [conceptnode,minimum width=4em,anchor=south west,fill=ublue!15,thin,align=left] (sec6box1) at ([xshift=0.5em,yshift=0.4em]sec6.south west) {\tiny{编码器-解码器}\\\tiny{框架}};
\node [conceptnode,minimum width=4em,anchor=west,fill=ublue!15,thin,align=left,minimum height=2em] (sec6box2) at ([xshift=0.2em]sec6box1.east) {\scriptsize{注意力机制}};
\node [conceptnode,minimum width=7.5em,anchor=west,fill=purple!15,thin,align=left] (sec6box3) at ([xshift=0.2em]sec6box2.east) {\tiny{基于RNN和Transformer}\\\tiny{的神经机器翻译建模}};
\node [conceptnode,minimum width=4em,anchor=west,fill=purple!15,thin,align=left,minimum height=2em] (sec6box4) at ([xshift=0.2em]sec6box3.east) {\scriptsize{训练与推断}};

\draw [->,very thick] ([yshift=0.2em]sec5.north) -- ([yshift=-0.2em]sec6.south);
\draw [->,very thick,dotted] ([yshift=0.2em,xshift=-2em]sec4.north) .. controls +(north:5.0em) and +(west:4em) .. ([xshift=-0.2em]sec6.west);
\draw [->,thick,dotted] ([xshift=3em,yshift=0.2em]pretrainbox.north) .. controls +(north:15em) and +(south:15em) .. ([xshift=0em,yshift=-0.0em]sec6box3.south);

% section 7
\node [secnode,anchor=south,purple,fill=white,minimum height=6.3em] (sec7) at ([yshift=3em]sec6.north) {};
\node [anchor=north] (sec7label) at (sec7.north) {\small{神经机器翻译实战}};
\node [anchor=north west,draw=purple,thick,fill=white,rounded corners] (sec7title) at ([xshift=-0.3em,yshift=0.3em]sec7.north west) {{\footnotesize\bfnew{\color{purple} 第七章}}};
\node [conceptnode,minimum width=4em,anchor=south west,fill=ublue!15,thin,align=left,minimum height=4.2em] (sec7box1) at ([xshift=0.5em,yshift=0.4em]sec7.south west) {\tiny{数据处理、}\\\tiny{子词切分}};

\node [anchor=north west,minimum width=5em,anchor=north west,fill=purple!15] (sec7box2) at ([xshift=0.5em,yshift=-0.2em]sec7box1.north east) {\tiny{正则化}};
\node [anchor=north west,minimum width=5em,anchor=north west,fill=purple!15] (sec7box3) at ([yshift=-0.1em]sec7box2.south west) {\tiny{增大模型容量}};
\node [anchor=north west,minimum width=5em,anchor=north west,fill=purple!15] (sec7box4) at ([yshift=-0.1em]sec7box3.south west) {\tiny{大批量训练}};

\node [anchor=north west,minimum width=5em,anchor=north west,fill=purple!15] (sec7box5) at ([xshift=0.6em]sec7box2.north east) {\tiny{推断优化}};
\node [anchor=north west,minimum width=5em,anchor=north west,fill=purple!15] (sec7box6) at ([yshift=-0.1em]sec7box5.south west) {\tiny{译文长度控制}};
\node [anchor=north west,minimum width=5em,anchor=north west,fill=purple!15] (sec7box7) at ([yshift=-0.1em]sec7box6.south west) {\tiny{多模型集成}};

\node [anchor=north west,minimum width=5em,anchor=north west,fill=purple!15] (sec7box8) at ([xshift=0.6em]sec7box5.north east) {\tiny{深层模型}};
\node [anchor=north west,minimum width=5em,anchor=north west,fill=purple!15] (sec7box9) at ([yshift=-0.1em]sec7box8.south west) {\tiny{知识精炼}};
\node [anchor=north west,minimum width=5em,anchor=north west,fill=purple!15] (sec7box10) at ([yshift=-0.1em]sec7box9.south west) {\tiny{单语数据使用}};

\node [draw,dotted,thick,inner sep=1pt] [fit = (sec7box2) (sec7box3) (sec7box4)] (trainbox) {};
\node [draw,dotted,thick,inner sep=1pt] [fit = (sec7box5) (sec7box6) (sec7box7)] (inferencebox) {};
\node [draw,dotted,thick,inner sep=1pt] [fit = (sec7box8) (sec7box9) (sec7box10)] (advancedbox) {};

\draw [->,very thick] ([yshift=0.2em]sec6.north) -- ([yshift=-0.2em]sec7.south);
\draw [->,very thick,dotted] ([yshift=0.2em,xshift=-3em]sec4.north) .. controls +(north:7.0em) and +(west:6em) .. ([xshift=-0.2em]sec7.west);

%caption
\node [anchor=north] (caption) at ([xshift=0.4em,yshift=-1em]sec1.south) {\footnotesize{本书各章节及核心概念关系图}};


\end{tikzpicture}

\end{figure}
%-------------------------------------------

\end{spacing}








