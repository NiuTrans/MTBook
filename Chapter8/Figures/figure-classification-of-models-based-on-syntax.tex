%%%------------------------------------------------------------------------------------------------------------
%%% 句法模型的分类
\begin{center}
\begin{tikzpicture}

\begin{scope}
\tikzstyle{cnode} = [minimum width=7.0em,minimum height=2.5em,rounded corners=0.2em,draw,thick];
\tikzstyle{xnode} = [minimum width=4.5em,minimum height=2.5em,rounded corners=0.2em,draw,thick];

\node[cnode,anchor=south,minimum width=10.0em,fill=green!25,align=center] (cat0) at (0,0) {\footnotesize{(广义上)}\\\footnotesize{基于句法的模型}};
\node[cnode,anchor=north,fill=red!25,align=left] (cat1) at ([xshift=-7.5em,yshift=-2em]cat0.south) {\footnotesize{基于形式文法}\\\footnotesize{的模型}};
\node[cnode,anchor=north,fill=blue!25,align=left] (cat2) at ([xshift=7.5em,yshift=-2em]cat0.south) {\footnotesize{基于语言学}\\\footnotesize{句法的模型}};


\node[xnode,anchor=north,fill=red!25,align=left] (itg) at ([xshift=-3.5em,yshift=-2.0em]cat1.south) {\footnotesize{反向转录}\\\footnotesize{文法}};
\node[xnode,anchor=north,fill=red!25,align=left] (hiero) at ([xshift=3.5em,yshift=-2.0em]cat1.south) {\footnotesize{层次短语}\\\footnotesize{模型}};
\node[xnode,anchor=north,fill=blue!25,align=left] (s2t) at ([xshift=-5.5em,yshift=-2.0em]cat2.south) {\footnotesize{串到树}\\\footnotesize{模型}};
\node[xnode,anchor=north,fill=blue!25,align=left] (t2s) at ([xshift=0.0em,yshift=-2.0em]cat2.south) {\footnotesize{树到串}\\\footnotesize{模型}};
\node[xnode,anchor=north,fill=blue!25,align=left] (t2t) at ([xshift=5.5em,yshift=-2.0em]cat2.south) {\footnotesize{树到树}\\\footnotesize{模型}};


\draw [-,thick] ([yshift=-0.1em,xshift=1em]cat0.south) -- ([xshift=-1.5em,yshift=0.1em]cat2.north);
\draw [-,thick] ([yshift=-0.1em,xshift=-1em]cat0.south) -- ([xshift=1.5em,yshift=0.1em]cat1.north);
\draw [-,thick] ([yshift=0.1em]itg.north) -- ([xshift=-0.5em,yshift=-0.1em]cat1.south);
\draw [-,thick] ([yshift=0.1em]hiero.north) -- ([xshift=0.5em,yshift=-0.1em]cat1.south);
\draw [-,thick] ([yshift=0.1em]s2t.north) -- ([xshift=-0.8em,yshift=-0.1em]cat2.south);
\draw [-,thick] ([yshift=0.1em]t2s.north) -- ([xshift=-0.0em,yshift=-0.1em]cat2.south);
\draw [-,thick] ([yshift=0.1em]t2t.north) -- ([xshift=0.8em,yshift=-0.1em]cat2.south);

\node [anchor=north] (itglabel) at (itg.south) {\scriptsize{(Wu, 1995)}};
\node [anchor=north] (hierolabel) at (hiero.south) {\scriptsize{(Chiang, 2005)}};
\node [anchor=north,align=left] (s2tlabel) at (s2t.south) {\scriptsize{(Galley et al.,}\\\scriptsize{\ 2004; 2006)}};
\node [anchor=north,align=left] (t2slabel) at (t2s.south) {\scriptsize{(Liu et al.,}\\\scriptsize{\ 2006)}};
\node [anchor=north,align=left] (t2tlabel) at (t2t.south) {\scriptsize{(Eisner, 2003)}};

\end{scope}

\end{tikzpicture}
\end{center}